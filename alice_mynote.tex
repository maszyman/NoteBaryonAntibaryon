\newpage
\section{Introduction}
\label{sec:overview}
In the analysis, we present the measurements of baryon-antibaryon correlations in Pb--Pb collisions at $\sqrt{s_{\mathrm{NN}}}=2.76$~TeV registered by the ALICE experiment. The method of two-particle correlations (commonly referred to as \emph{femtoscopy})  allows for extracting the space-time characteristics of the emitting source created in heavy-ion collision. Using this technique one can also attempt to extract paramters of strong interaction~\cite{}.

\section{Data analysis}
\label{sec:analysis}
\subsection{Data sample}
\subsubsection{Data selection}

\subsubsection{Monte Carlo}

\subsection{Event selection}
\subsection{Particle identification}
\subsubsection{(Anti-)protons identification}
\subsubsection{(Anti-)lambdas identification}


\subsection{Track selection}

\subsection{Pair selection}

\subsection{Proton and lambda fraction with respect to their origin}
In order to estimate fraction of protons and lambdas as a   MC Hijing LHC12a17a fix
 Fractions checked before and after reconstruction
 Kinematic cuts the same as in data
 Before reconstruction significant contribution of protons at low $p_{T}$  with PDG of mothers corresponding to $\pi$, $K^0_L$, $K^0_S$, $K^+$, $D^+_S$, $J/\psi$, $B^0$, $B^+$, $B^0_S$ - coming from interactions with material?
 Cross-check with Therminator2 (no reconstruction!)


\section{Results}

\subsection{Correlation functions}

\subsection{Fitting procedure}
\subsubsection{\pap~theoretical function}

\subsection{Systematic uncertainties}
\begin{itemize}
\item non-femtoscopic background
\item fractions (Hijing vs. Therminator)
\item number of secondaries from material
\item momentum resolution correction
\item ALICE magnetic fields ++ vs. - - 
\item PID
\item different scenarios for interaction parameters
\item DCA templates
\item \pal~vs.~\apl
\item fitting procedure    
\end{itemize}

\subsubsection{Momentum resolution}
Correction for momentum resolution is taken into account in the fitting procedure. Fit function is smeared with a gaussian function with the width corresponding the momentum resolution for the pairs of interest. Following formula is used:
\begin{equation}
C_c(q_c) = \int_{-3\sigma}^{+3\sigma}C_{th}(q_c-q) Gaus(q_t, \sigma)(q) |q_c-q|^2 d q,
\end{equation}
where $C_c$ is the corrected function, $C_{th}$ is the ideal function, $\sigma$ is the momentum resolution.

%% \section{Summary}
%% \label{sec:summary}
% In summary, correlations of all combinations of pairs of protons and antiprotons have been measured in Pb--Pb collisions at $\sqrt{s_{\mathrm{NN}}}=2.76$~TeV in the ALICE experiment. The femtoscopic parameters for the radius of the proton source are extracted from one-dimensional pp, $\bar{\mathrm{p}}\bar{\mathrm{p}}$ and p$\bar{\mathrm{p}}$ correlation functions. The fit includes final-state interactions and quantum statistics for identical pairs of (anti)protons. The fit takes into account residual correlations coming from p$\Lambda$ system. Two-proton correlations show an increase of the radius with increasing multiplicity and slight decrease of the radius with increasing  pair transverse momentum. %Also, transverse mass scaling is observed between $\pi\pi$, K$^{\mathrm{ch}}$K$^{\mathrm{ch}}$, \Kzs\Kzs~and two-proton invariant radii.

\begin{thebibliography}{99}

\end{thebibliography}
